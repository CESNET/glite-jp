\section{Running and stopping the services}
\TODO{}


\subsection{JPPS}


\subsection{JPIS}

\begin{alltt}
Preferred way of starting the daemon is using start-up script
(config/startup). It loads glite.conf file (personal version may be stored
in ~/.glite.conf) where many variables may be set to configure the daemon.
The script takes following variables:

GLITE_JPIS_CONFIG       - server config file specification
                        (default is $GLITE_LOCATION//etc/glite-jpis-config.xml)
GLITE_JPIS_DEBUG        - setting to '-d' forces the daemon not to daemonize
GLITE_JPIS_QT           - defines query type
                        'hist' ... history query
                        'cont' ... continuous query
                        'both' ... combination of previous types
GLITE_JPIS_AUTH         - setting to '-n' forces the daemon not to check
                        authorisation
GLITE_JPIS_PORT         - used port (default 8902)
GLITE_JPIS_DB           - database connection string
                        (default jpis/@localhost:jpis)
GLITE_JPIS_LOGFILE      - log file
                        (default is $GLITE_LOCATION_VAR/log/glite-jp-indexd.log)
GLITE_JPIS_PIDFILE      - pid file
                        (default is $GLITE_LOCATION_VAR/run/glite-jp-indexd.log)

\end{alltt}
