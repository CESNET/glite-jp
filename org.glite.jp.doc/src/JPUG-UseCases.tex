\section{Job Provenance use cases}

\subsection{Prerequisities}

\subsubsection{LB/JP relationship}
When JP deployed, any job in a terminal state will disappear from LB
after preconfigured timeout (one week for example). If a user wants
any information about such a job before this timeout (or before it
reach a terminal state) he must use the LB service (please refer to LB
user's guide). After that timeout he must use the JP service.

% TODO: update
% For LB configuration please see gLite installation guide. For a
% technical description of LB-JP interactions please see
% \texttt{http://egee.cesnet.cz/en/JRA1/LB-JP-interaction-guide.pdf}.

\subsubsection{JP service location}
To call JP you need to know JP services address. There are two services:
\begin{itemize}
\item JP primary storage (JPPS)\\
  From JP design point of view there are only few PS in the
  grid. Expected implementation is that these JPPS locations
  are preconfigured in a UI instance while one of them is configured as
  default JPPS.
\item JP index server (JPIS)\\
  Each index server is build (configured and started) by site/VO/user
  group administrator (or even "senior user") based on given
  community needs (expected queries and its optimization). So in
  principle the index server location for a given query is to be
  provided by the user. We expect that the UI instance will provide
  mechanism allowing selection from preconfigured JPIS servers list.
\end{itemize}

\subsection{JP use case 1 -- get job info}

The scenario:
\begin{itemize}
\item The user wants information about a particular job. He knows a
  job id. Job isn't longer in the LB. Procedure: Ask the JPPS to get all
  or selected attributes of job.
\end{itemize}

The implementation:
\begin{itemize}
 \item Let a user to specify attributes to be returned. See section
  \ref{attributes}.
 \item Call GetJobAttributes operation of a JPPS and display the values
  returned.
\end{itemize}

Examples and hints:
\begin{itemize}
 \item \texttt{org.glite.jp.primary/examples/jpps-test.c}\\
  This utility is used for all JPPS operations. Some hints how to use it
  can be find in the test plan document.
\end{itemize}

\subsection{JP use case 2 -- get job files}

The scenario:
\begin{itemize}
 \item The user knows a job id, job is in a terminal state. The user wants
  all files (LB event dump, sandbox) stored by JP for futher processing.
\end{itemize}

The implementation:
\begin{itemize}
 \item Call GetJobFiles operation of a JPPS. You will get a list of URLs
  which can be used to download the files.
\end{itemize}

Examples and hints:
\begin{itemize}
 \item The same as use case 1.
\end{itemize}

\subsection{JP use case 3 -- job lookup}

The scenario:
\begin{itemize}
 \item The user is looking for jobs with specific properties. In this case
  (no job id known) a JPIS must be used. There are the same query interface
  provided by any JPIS but if a particular query can be answered by
  the given JPIS depends on its configuration (configuration
  determines which attributes are uploaded by PS to IS, and which of
  them are indexed).
 \item The user should know the proper JPIS to use for its particular
  needs.
 \item The scenario can continue by the JP use cases number 1 and 2 described
  above (JPIS answer will contain job ids and identification of JPPSs
  to ask for all available JP data about the jobs).
\end{itemize}

The implementation:
\begin{itemize}
 \item The user will select a JPIS and provide query. The JPIS operation
  QueryJobs is called and list of jobs matching the query is returned.
\end{itemize}

Examples and hints:
\begin{itemize}
 \item JPIS CLI tool\\
   org.glite.jp.index/examples/jpis-client.c

 \item example in org.glite.jp.index/examples/jpis-test.c (starting
   from line 161)
\end{itemize}

\subsection{JP use case 4 -- job annotation}

The scenario:
\begin{itemize}
 \item The user wants to add a user tag (annotation) to a job. He must know
  the job id(s) (or use the JP use case number 3 to find it).
\end{itemize}

The implementation:
\begin{itemize}
 \item Call RecordTag operation of JPPS for the job(s) to add requested
  user tag.
\end{itemize}

Examples and hints:
\begin{itemize}
 \item The same as use case 1.
\end{itemize}


\subsection{Job attributes}
\label{attributes}
Job attributes are referenced by its names. Each attribute belongs to
one namespace (represented by a prefix in the attribute name).

A namespace is defined by a service (currently we have one for LB and
one for JP) providing its data to the JP or a user group/experiment
who wants to attach its own data to the job.

It is expected that UI have preconfigured list of available namespaces
and XML schema for each namespace (the schema can be automatically
retrieved based on the namespace name). A list of available attributes
is generated from these schemas when user have to select attributes to
be retrieved from JP.

\begin{itemize}
 \item The namespaces (schema is available at the URL representing namespace):\\
  http://egee.cesnet.cz/en/Schema/LB/Attributes\\
  http://egee.cesnet.cz/en/Schema/JP/System   <<<<<<<(NOT YET)\\
 \item There are header files with known names of attributes generated from
  these schema files in our build procedure:\\
  org.glite.lb.server/build/jp\_job\_attrs.h\\
  org.glite.jp.common/interface/known\_attr.h  <<<<<<<\\
\end{itemize}


\subsection{Authentication and authorization}
All the calls must be authenticated by user credentials. In the
current JP release only implicit ACLs are available -- the job
information is available for job owner only.

