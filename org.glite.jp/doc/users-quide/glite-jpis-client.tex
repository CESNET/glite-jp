% 
% -------------------------------------------------------------
% Refentry 
% ------------------------------------------------------------- 	
\section*{glite-jpis-client}
\label{glitejpisclient}\hypertarget{glitejpisclient}{}%
\label{name}

%\section*{Nom}
glite-jpis-client --- client interface for JP IS\label{synopsis}
\subsection*{Synopsis}
\label{id2455104}
\begin{list}{}{\setlength{\itemindent}{-\leftmargin}\setlength{\parsep}{0mm}}
\item\raggedright\texttt{glite-jpis-client [ -h | --help ] [ -i | --index-server  \textit{JPIS:PORT}] [ -q | --query-file  \textit{IN\_FILE.XML}] [ -t | --test-file  \textit{IN\_FILE.XML}] [ -e | --example-file  \textit{OUT\_FILE.XML}] [ -f | --format   {xml | human}]}
\end{list}

\subsection*{DESCRIPTION}
\label{id2417458}

{\bfseries{glite-jpis-client}} is command line interface for querying the Job Provenance Index Server. It takes the XML input, process the QueryJobs operation and returns the result in specified format.

\subsection*{OPTIONS}
\label{id2417597}

With no options you get simple usage message as with {\texttt{{-h}}}.

\begin{description}
% \null and \mbox are tricks to induce different typesetting decisions
\item[{{\texttt{{-h}}}\docbooktolatexpipe{}{\texttt{{--help}}}}]\null{}
Displays usage message.
% \null and \mbox are tricks to induce different typesetting decisions
\item[{{\texttt{{-i}}}\docbooktolatexpipe{}{\texttt{{--index-server}}}}]\null{}
Specifies Job Provenance Index Server as {\ttfamily\itshape{{HOST:PORT}}}.
% \null and \mbox are tricks to induce different typesetting decisions
\item[{{\texttt{{-q}}}\docbooktolatexpipe{}{\texttt{{--query-file}}}}]\null{}
Process the QueryJobs operation. Requires input data in file {\ttfamily\itshape{{IN\_FILE.XML}}}, for using stdin specify {\texttt{{-}}}.

The input and output data are in XML format with XSD schema, which can be found in {\texttt{{JobProvenanceISClient.\dbz{}xsd}}} (element QueryJobs for input and QueryJobsResponse for output).
% \null and \mbox are tricks to induce different typesetting decisions
\item[{{\texttt{{-t}}}\docbooktolatexpipe{}{\texttt{{--test-file}}}}]\null{}
Test the input data from {\ttfamily\itshape{{IN\_FILE.XML}}} (or from stdin if {\texttt{{-}}} is specified) and prints the found content.
% \null and \mbox are tricks to induce different typesetting decisions
\item[{{\texttt{{-e}}}\docbooktolatexpipe{}{\texttt{{--example-file}}}}]\null{}
Write the example input data to file {\ttfamily\itshape{{OUT\_FILE.XML}}} (or to stdout if {\texttt{{-}}} is specified). The XML is valid against XSD schema in {\texttt{{JobProvananceISClient.\dbz{}xsd}}}, formating may vary according to used gsoap version.
% \null and \mbox are tricks to induce different typesetting decisions
\item[{{\texttt{{-f}}}\docbooktolatexpipe{}{\texttt{{--format}}}}]\null{}
Use {\ttfamily\itshape{{FORMAT}}} as output format type. You can specify {\texttt{{xml}}} for interchangeable XML output or {\texttt{{human}}} for nice looking human readable output.
\end{description}
\noindent 
\subsection*{RETURN VALUE}
\label{id2417767}

\begin{description}
% \null and \mbox are tricks to induce different typesetting decisions
\item[{0}]\null{}
Success.
% \null and \mbox are tricks to induce different typesetting decisions
\item[{-1}]\null{}
Communication error or error from the remote server.
% \null and \mbox are tricks to induce different typesetting decisions
\item[{EINVAL}]\null{}
In most cases XML parsing error.
% \null and \mbox are tricks to induce different typesetting decisions
\item[{other error}]\null{}
Other error from errno.
\end{description}
\noindent 
\subsection*{EXAMPLES}
\label{id2417822}

\begin{description}
% \null and \mbox are tricks to induce different typesetting decisions
\item[{{\bfseries{glite-jpis-client --example-file query.xml}}}]\null{}
Save the example query parameters to file {\texttt{{query.\dbz{}xml}}}.
% \null and \mbox are tricks to induce different typesetting decisions
\item[{{\bfseries{glite-jpis-client --query-file query.xml}}}]\null{}
Queries the local index server running on default port with the query parameters specified in the file {\texttt{{query.\dbz{}xml}}}.
% \null and \mbox are tricks to induce different typesetting decisions
\item[{{\bfseries{glite-jpis-client -i localhost:8902 -q - -f human}}}]\null{}
Queries the index server running on local host on the port 8902 with the query parameters from stdin and show results in non-XML form.
\end{description}
\noindent 


\subsection*{SEE ALSO}
\label{id2418102}

glite-jp-indexd(8)

\subsection*{AUTHOR}
\label{id2418112}

EU DataGrid Work Package 1, CESNET group.
