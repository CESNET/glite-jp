\documentclass{egee}
\usepackage{comment}

\def\LB{L\&B}

\title{Job Provenance Test Plan}
\author{CESNET EGEE JRA1 team}
\DocIdentifier{EGEE-JRA1-??}
\Date{\today}
\Activity{JRA1: Middleware Engineering and Integration}
\DocStatus{DRAFT}
\Dissemination{PUBLIC}
\DocumentLink{}

\def\req{\noindent\textbf{Prerequisities:}}
\def\how{\noindent\textbf{How to run:}}
\def\result{\noindent\textbf{Expected result:}}

\def\path#1{{\normalfont\textsf{#1}}}
\def\code#1{\texttt{#1}}

\def\todo#1{\textbf{TODO:} #1}


\specialcomment{hints}{\par\noindent\textbf{Hints: }\begingroup\slshape}{\endgroup}
%\includecomment{hints}

\begin{document}

\begin{center}
{\bf Delivery Slip}
\end{center}
\begin{tabularx}{\textwidth}{|l|l|l|X|X|}
\hline
           & {\bf Name} & {\bf Partner} & {\bf Date} & {\bf Signature} \\
\hline
{\bf From} &                  &  & & \\
\hline
{\bf Reviewed by} & &  & & \\

\hline
{\bf Approved by} & & & & \\
\hline
\end{tabularx}

\begin{center}
{\bf Document Change Log}
\end{center}

\begin{tabularx}{\textwidth}{|l|l|X|X|}
\hline
{\bf Issue } & {\bf Date  } & {\bf Comment } & {\bf Author  } \\   \hline

\hline
\end{tabularx}

\begin{center}
{\bf Document Change Record}
\end{center}

\begin{tabularx}{\textwidth}{|l|l|X|}
\hline
{\bf Issue } & {\bf Item  } & {\bf Reason for Change } \\   \hline

\hline
\end{tabularx}

%
% Official text received on October 6, 2004
%
\vfill{\bf Copyright }\copyright{\bf Members of the EGEE Collaboration. 2004. 
See http://eu-egee.org/partners for details on the copyright holders. 

EGEE (``Enabling Grids for E-science in Europe'') is a project funded by
the European Union.  For more information on the project, its partners
and contributors please see http://www.eu-egee.org.

You are permitted to copy and distribute verbatim copies of this
document containing this copyright notice, but modifying this document
is not allowed. You are permitted to copy this document in whole or in
part into other documents if you attach the following reference to the
copied elements: ``Copyright }\copyright{\bf 2004. Members of the EGEE
Collaboration. http://www.eu-egee.org''

The information contained in this document represents the views of
EGEE as of the date they are published. EGEE does not guarantee that
any information contained herein is error-free, or up to date.

EGEE MAKES NO WARRANTIES, EXPRESS, IMPLIED, OR STATUTORY, BY
PUBLISHING THIS DOCUMENT.}


\clearpage

\newpage
\tableofcontents
\newpage

\section{Rationale}
\subsection{Glossary}
\begin{itemize}
\item JP -- Job Provenance
\item JPPS -- Job Provenance Primary Storage
\item JPPSBE -- Job Provenance Primary Storage Back-end
\item JPIS -- Job Provenance Index Server
\end{itemize}

\todo{}

\section{Test Coverage}
\todo{}

%\chapter{Test Cases}

\section{JPPS standalone tests}

\subsection{Job registration}

\subsubsection{Basic functionality}
- call RegisterJob
* call GetJobAttributes owner to verify

\subsubsection{AuthZ check}
* call GetJobAttributes with different credentials - should fail

\subsection{Tag recording}
- call RecordTag
* call GetJobAttributes to verify
- record and retrieve more values of the same tag

\subsection{File upload}

\subsubsection{Basic functionality}
- call StartUpload, LB dump file type
* check with GetJobFiles -- shoud return nothing
- upload via ftp
- call CommitUpload
* check with GetJobFiles -- should return URL
- retrieve and check the file

\subsubsection{AuthZ checks}
(should fail)
* call GetJobFiles with different credentials

* StartUpload with different credentials

- StartUpload
* ftp upload with different credentials

* ftp GET with different credentials

\subsubsection{Cleanup}
(Foreseen test for feature which is not implemented yet)
- call StartUpload, short timeout
- upload via ftp
(don't call CommitUpload)
* uploaded file should be purged after timeout

\section{LB plugin}
\todo{TBD}

\section{JPPS-JPIS interaction (feeds)}

\todo{import this chapter from testplan.txt}

set of queries (how many?) with different "triggering conditions":
- on job registration
- on LB file upload
- on RecordTag

corresponding sets of jobs to each query, each containing jobs which match
and which don't

- initial IS release -- single query, so just one set of jobs
- due to 3.2 no point in pre-loading PS database, use 1.3.1

\subsection{Single batch feed}
- upload jobs to PS
- start feed
* check IS contents (jobs and expected attr values)

\subsection{single incremental feed}
- register feed
- upload jobs to PS one by one
* check IS contents (matching jobs should turn up, others not)

\subsection{Multiple feeds at time}
TODO

\subsection{Advanced feed features (to be implemented)}
- remove (not implemented in PS yet)
- splitted info about one job (check that the PS doesn't duplicate
  attribute values) - probably covered in 3.2


\subsection{PS-IS AuthZ}
TODO, if any

\section{IS queries}
\todo{Import from org.glite.jp.index/doc/README file}


TBD: insert job sets via JP-IS interaction or directly?
        - better to populate database directly, independent on previous chain

All basic tests:
- clear IS database
- insert prepared job set
- ask queries and check answers
- clear database

TBD: Is one job set enough?
        - better to have one complete set

\subsection{Simple query}
- using CLI

\subsection{Extended query}
- using CLI

\subsection{Check "origin" behaviour (not implemented yet)}
- queries with origin tag

\subsection{AuthZ checks}
- non owner queries should fail
- to be implemented: ACLs and its evaluation

\subsection{IS CLI}
- standalone tests?     - not now
  - prepared config files and command line parameters
  - check expected QueryJobs contents

\section{IS standalone advanced features}
\todo{To be implemented}

\subsection{Server startup}

\subsubsection{Reboot persistency / configuration vs. database content}
    situations handling
- prepared config files
- checking behaviour (how?) after reboot with different config file

\subsubsection{Registration of PS feeds}
! already covered by 3
- prepared config files
- checking appropriate FeedIndex calls

\subsection{Admin interface}
\todo{Admin interface not implemented yet}

\subsection{Type plugin}
\todo{type plugin tests -- to be designed, future type plugin implementation}

\section{Deployment}
\todo{tests on JP deployment process}
\todo{TBD}



\end{document}
